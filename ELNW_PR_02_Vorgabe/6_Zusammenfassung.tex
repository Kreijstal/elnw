%---------------------------------%
%----- 6_Zusammenfassung.tex -----%
%---------------------------------%
%---------------------------------%
%
%------------------------------%
%----- Beginn eures Teils -----%
%------------------------------%
%
Das Wichtigste ist zu erkennen, was die Bedeutung der Ordnung einer Schaltung bedeutet und wie sie der Ordnung der DGL entspricht, die wir lösen müssten, um den Zustand der Schaltung zu jedem Zeitpunkt zu beschreiben. Wir lernen auch Ausgleichvorgänge kennen, bei denen die Schaltkreise von einem Zustand in einen anderen übergehen, aber dank Kondensatoren oder Induktivitäten von einem Zustand in den anderen "glatt" übergehen. Wir lernen auch, dass die Theorie immer ein bisschen unpassend sein wird als das, was wir tatsächlich messen werden.



%
%
%
\begin{flushright}
  \textit{\autorA}
\end{flushright}
%
%------------------------------%
%------ Ende eures Teils ------%
%------------------------------%
%
%
%