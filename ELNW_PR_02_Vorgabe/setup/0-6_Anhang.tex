%--------------------------------------------------------%
%-------------------- 0-6_Anhang.tex --------------------%
%--------------------------------------------------------%
%----- Hier könnt ihr weiterführende Informationen, -----%
%----- umfangreiche Codelisten, Messdatensammlungen -----%
%----- oder andere Quellmaterialien einfügen. -----------%
%--------------------------------------------------------%
%--------------------------------------------------------%
%----- Achtung: Fügt bitte ausschließlich Inhalte -------%
%-----          ein, die wirklich relevant für eure -----%
%-----          Ausführungen sind! ----------------------%
%--------------------------------------------------------%
%--------------------------------------------------------%
%
\section{Verwendeter Matlab-Code}
\label{app:Matlab}
%
Code zum Berechnung der Tabelle \ref{tab:6_GEMerzeught}
\begin{minted}[mathescape,
               linenos,
               numbersep=3pt,
               gobble=0,
               frame=lines,
               framesep=2mm,breaklines]{matlab}
messwerte=load('ELNW_PR_01_Vorgabe_Messwerte.mat');
%das ganze hier ist um , als Dezimaltrenner im Matlab auszudrücken
df=java.text.DecimalFormat("##0.#####E0", ...
    java.text.DecimalFormatSymbols.getInstance(java.util.Locale.GERMAN));
df.setMaximumFractionDigits(3);
dfs=java.text.DecimalFormat("0", ...
    java.text.DecimalFormatSymbols.getInstance(java.util.Locale.GERMAN));
dfs.setMaximumFractionDigits(3);
fi = @(varargin)varargin{length(varargin)-varargin{1}};
paren = @(x, varargin) x(varargin{:});
curly = @(x, varargin) x{varargin{:}};
asd = @(x, varargin) vertcat(x{:});
nformat=@(num)(string(fi(log10(num)>4||log10(num)<-2.6,df.format(num), ...
    dfs.format(num))));
everyarr= @(lol) asd(arrayfun(nformat,lol,"UniformOutput",false));
messwerte2=messwerte;
messwerte2.dt=messwerte2.dt*1000000;
neumess=structfun(everyarr,messwerte2,'UniformOutput',false);
writetable(struct2table(neumess),"messwertelab1.txt" ,'Delimiter',';')
R=470;
messwerte.phi=messwerte.dt.*messwerte.f.*2*pi;
messwerte.Ipp=messwerte.U_Rpp./R;
messwerte.z=(((messwerte.U_Epp-messwerte.U_Rpp))./((messwerte.Ipp))+R);
messwerte.y=1./messwerte.z;
messwerte.rez=messwerte.z.*cos(-messwerte.phi);
messwerte.imz=messwerte.z.*sin(-messwerte.phi);
messwerte.rey=messwerte.y.*cos(messwerte.phi);
messwerte.imy=messwerte.y.*sin(messwerte.phi);
writetable(table(everyarr(messwerte.phi), ...
    everyarr(messwerte.phi.*(360/(2*pi))), ...
    everyarr(messwerte.Ipp*1000), ...
    everyarr(messwerte.z./1000), ...
    everyarr(messwerte.y.*1000), ...
    everyarr(messwerte.rez./1000), ...
    everyarr(messwerte.imz./1000), ...
    everyarr(messwerte.rey.*1000), ...
    everyarr(messwerte.imy.*1000),'VariableNames', ...
    {'phi in rad','phi in grad','Ipp in mA', ...
    '|z| in kOhm','|y| in mS','Re(z) in kOhm', ...
    'Im(z) in kOhm','Re(y) in mS','Im(y) in mS'}), ...
    "messwertelab1p.txt" ,'Delimiter',';');
\end{minted}
Code für plotten
\begin{minted}[mathescape,
               linenos,
               numbersep=3pt,
               gobble=0,
               frame=lines,
               framesep=2mm,breaklines]{matlab}
simulierte=readLTspice("./1lab1.txt","Nyquist")
figure(2)
clf
plot(simulierte(:,4)*1e3,simulierte(:,5)*1e3,"r")
grid on
figure(1)
clf
hold on
subplot(1,2,2);
hold on
plot(simulierte(:,2),simulierte(:,3)*1e-3)
subplot(1,2,1);
hold on
plot(simulierte(:,2),simulierte(:,3)*1e-3)
figure(1)
subplot(1,2,2);
plot(messwerte.rez,messwerte.imz*1e-3,"c x")
subplot(1,2,1);
plot(messwerte.rez,messwerte.imz*1e-3,"c x")
figure(2)
hold on
plot(messwerte.rey.*1e3,messwerte.imy.*1e3,"m x")
figure(2)
legend('simulierte Daten','Messwerte')
title("Admitanz der RC-Reihenschaltung")
xlabel("Re(Z) im mS")
ylabel("Im(Z) im mS")
figure(1)
legend('simulierte Daten','Messwerte')
legend('Location','southoutside')
subplot(1,2,1);
grid on
title("Impedanz der RC-Reihenschaltung")
xlim([0 470*2])
xlabel("Re(Z) im \Omega")
ylabel("Im(Z) im k\Omega")
subplot(1,2,2);
title("Impedanz - von 0 bis -2k\Omega")
grid on
xlabel("Re(Z) im \Omega")
ylabel("Im(Z) im k\Omega")
xlim([0 470*2])
ylim([-2.5 0])
hold off
\end{minted}
%
%
%
% Auch im Anhang sollte die Autorenschaft gekennzeichnet werden. Ihr könnt aber im Falle von Scilab-Code die Autorenschaft im Code selbst kennzeichnen.
%
\begin{flushright}
  \textit{\autorA}
\end{flushright}
%
%
%
%\newpage
%\section{Datenblätter}
%\label{app:Datenblätter}
%
%\lipsum[7-9]
%
%
%
% Auch wenn es sich nur um Abbildungssammlungen handelt, muss die Autorenschaft trotzdem gekennzeichnet werden.
%
%\begin{flushright}
  %\textit{\autorA}
%\end{flushright}
%
%
%