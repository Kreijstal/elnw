%--------------------------------------------------------%
%-------------------- 0-6_Anhang.tex --------------------%
%--------------------------------------------------------%
%----- Hier könnt ihr weiterführende Informationen, -----%
%----- umfangreiche Codelisten, Messdatensammlungen -----%
%----- oder andere Quellmaterialien einfügen. -----------%
%--------------------------------------------------------%
%--------------------------------------------------------%
%----- Achtung: Fügt bitte ausschließlich Inhalte -------%
%-----          ein, die wirklich relevant für eure -----%
%-----          Ausführungen sind! ----------------------%
%--------------------------------------------------------%
%--------------------------------------------------------%
%
\section{Verwendeter Matlab-Code}
\label{app:Matlab}
%
\begin{minted}[mathescape,
               linenos,
               numbersep=3pt,
               gobble=0,
               frame=lines,
               framesep=2mm,breaklines]{matlab}
t=readLTspice("./labor2.txt","Nyquist")
figure(1)
clf
subplot(2,1,1);
plot(t(:,1)*1e3,t(:,2))
hold on
plot(t(:,1)*1e3,t(:,3))
plot(t(:,1)*1e3,t(:,4))
grid on
xlabel("t in ms")
ylabel("U in V")
title("Spannungen der RC-Parallelschaltung" )
legend('U_E','U_{C_1}',"U_{C_2}")
hold off
subplot(2,1,2);
plot(t(:,1)*1e3,t(:,5)*1e3)
hold on
plot(t(:,1)*1e3,t(:,6)*1e3)
plot(t(:,1)*1e3,t(:,7)*1e3)
xlabel("t in ms")
ylabel("I in mA")
title("Strömungen der RC-Parallelschaltung" )
legend('I_{U_E}','I_{C_1}',"I_{C_2}")
grid on
messwerte=load('ELNW_PR_02_Vorgabe_Messwerte.mat')
messwerte.I_ges=(messwerte.u_E-messwerte.u_C)
R = 1e3;
C = 100e-9;
%für Idealität
t_ideal = [-0.1 0 0 0.5 0.5 1]
uE_ideal = [0 0 10 10 0 0];

% die ideale Aufladevorgang 
t1 = (0:0.01:0.5)*1e-3;                     
U_E1 = 10;

uC_ideal_EIN = U_E1 * (1-exp(-1/(R*C) * t1));

% Entladevorgang
t2 = (0.5:0.01:1) *1e-3;                  
t0 = 0.5e-3;                               
U_E2 = 10;                               

uC_ideal_AUS = U_E2 * exp(-1/(R*C) * (t2-t0));


tC_ideal = [t1 t2]*1e3;                 
uC_ideal = [uC_ideal_EIN uC_ideal_AUS];


uR_ideal_AUS = 0 - uC_ideal_AUS;
uR_ideal_EIN = 10 - uC_ideal_EIN;
uR_ideal = [uR_ideal_EIN uR_ideal_AUS];
i_ideal = uR_ideal;


figure(2)
clf
%plot(messwerte.t_mess,'DisplayName','t_mess');
hold on;
plot(messwerte.t_mess,messwerte.u_C,'bx','DisplayName','u_C');
plot(tC_ideal,uC_ideal,'b-','DisplayName','u_C Ideal');

plot(messwerte.t_mess,messwerte.u_E,'rx','DisplayName','u_E');
plot(t_ideal,uE_ideal,'r-','DisplayName','u_E Ideal');
ylabel("U in V")
yyaxis right
plot(messwerte.t_mess,messwerte.I_ges,'yx','DisplayName','I_{ges} U_R');
plot(tC_ideal,uR_ideal,'y-','DisplayName','I_{ges} U_R Ideal');
ylabel("I in mA")
legend()
%legend("u_C","u_E","I_{ges}, U_R")
xlabel("t in ms")
title("Messwerte für RC-Schaltung")
hold off;
\end{minted}

\section{Verwendeter Python-Code}
\label{app:Python}
\inputminted{python}{src/reader.py}
%
%
%
% Auch im Anhang sollte die Autorenschaft gekennzeichnet werden. Ihr könnt aber im Falle von Scilab-Code die Autorenschaft im Code selbst kennzeichnen.
%
\begin{flushright}
  \textit{\autorA}
\end{flushright}
%
%
%
%\newpage
%\section{Datenblätter}
%\label{app:Datenblätter}
%
%\lipsum[7-9]
%
%
%
% Auch wenn es sich nur um Abbildungssammlungen handelt, muss die Autorenschaft trotzdem gekennzeichnet werden.
%
%\begin{flushright}
  %\textit{\autorA}
%\end{flushright}
%
%
%