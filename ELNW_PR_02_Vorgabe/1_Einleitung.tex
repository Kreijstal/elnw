%----------------------------%
%----- 1_Einleitung.tex -----%
%----------------------------%
%----------------------------%
%
%\lipsum[2-4]
Ein elektrisches Netz im stationären Zustand kann mit komplexen Wechselstromrechnungen einfach und effizient analysiert
werden. Im Gegensatz dazu befinden sich die Zustandsvariablen Strom und Spannung im Allgemeinen während eines
Ausgleichsvorgangs 
nicht periodisch und muss mit Differentialgleichungen (DGL) beschrieben werden. Die Lösung ist
mathematisch anspruchsvoll und viel komplexer. In diesem Experiment soll der Ausgleichsvorgang beim Anschließen oder
Abschalten der Spannungsquelle an einer RC-Reihenschaltung untersucht werden. Zu diesem Zweck sollten im ersten Schritt die
theoretisch erwarteten Verläufe der Zustandsvariablen ermittelt werden. Im zweiten Schritt soll die Schaltung aufgebaut und die
Ausgleichsvorgänge aufgezeichnet werden. Im dritten Schritt soll die Schaltung in LTspice aufgebaut und simuliert werden, um
idealisierte Abläufe der Ausgleichsvorgänge zu erhalten. Zusätzlich soll im vierten Schritt die Schaltung um ein weiteres RC-
Element erweitert und erneut simuliert werden, um den Einfluss zusätzlicher RC-Elemente untersuchen zu können. Abschließend
sind die theoretischen, gemessenen und die aus der Simulation erhaltenen Ergebnisse zu beschreiben, zu vergleichen und zu
interpretieren.
%
%
% Bitte denkt daran, eure Autorenschaft namentlich zu kennzeichnen! Das gilt für jeden (Unter-)Abschnitt, den ihr bearbeitet habt.
%
\begin{flushright}
  \textit{\autorA}
\end{flushright}
%
%
%