\documentclass{scrartcl}
\usepackage{amsmath}
\usepackage{amsfonts}
\usepackage{tabto}
\usepackage{enumerate}
\usepackage[ngerman]{babel}
\usepackage[T1]{fontenc}
\usepackage{lmodern}
\usepackage{cancel}
\usepackage[utf8]{inputenc}
\usepackage{tabularx}
\usepackage[paper=a4paper,left=25mm,right=25mm,top=20mm,bottom=20mm]{geometry}
\usepackage{nccmath}
\usepackage{xcolor}
\usepackage{ragged2e}
%\usepackage{graphics}
\usepackage{graphicx}
\usepackage{fancyhdr}
\usepackage{mathtools}
\usepackage{float}
\usepackage{physics}
\usepackage[linkcolor=blu,colorlinks=false]{hyperref}

%\usepackage{breqn}
\usepackage{cfr-lm}
\usepackage{newtxmath}
\usepackage{cleveref}
\renewcommand{\headrulewidth}{0.5pt}
\renewcommand{\footrulewidth}{0.5pt}
\usepackage{microtype}  % Minature font tweaks
%\usepackage{cmathbb}
%\DeclareSymbolFont{eulerletters}{U}{zeur}{b}{n}
%\DeclareSymbolFont{cmrot1italics}{OT1}{cmr}{m}{it}
%\DeclareMathSymbol{\imag}{\mathord}{eulerletters}{`i}
%\DeclareMathSymbol{\imag}{\mathord}{cmrot1italics}{`i}
\DeclareSymbolFont{uiletters}{OT1}{cmr}{m}{ui}
\DeclareMathSymbol{\imag}{\mathalpha}{uiletters}{`i}
% Common shortcuts
\def\mbb#1{\mathbb{#1}}
\def\mfk#1{\mathfrak{#1}}
%\def\imag{\mathsf{i}}
\def\bN{\mbb{N}}
\def\bC{\mbb{C}}
\def\bR{\mbb{R}}
\def\bQ{\mbb{Q}}
\def\bZ{\mbb{Z}}
\newcommand{\func}[3]{#1\colon#2\to#3}
\newcommand{\vfunc}[5]{\func{#1}{#2}{#3},\quad#4\longmapsto#5}
\newcommand{\floor}[1]{\left\lfloor#1\right\rfloor}
\newcommand{\ceil}[1]{\left\lceil#1\right\rceil}
%Abstand zwischen Absätzen, Zeilenabstände
\voffset26pt
\parskip6pt
%\parindent1cm  %Rückt erste Zeile eines neuen Absatzes ein
\usepackage{setspace}
\onehalfspacing
\title{TET Hausaufgabe}
\author{Juan Pardo Martin (397882)}
\let\*\cdot
\let\-\rightarrow
\let\_\Rightarrow
\let\>\Leftrightarrow
\definecolor{l}{rgb}{0.0, 0.5, 0.0}
\let\c\textcolor
%\mathbb{R}
%\begin{align*}
\titlehead
{
\begin{tabular}{ll}
\begin{minipage}{0.5\textwidth}
%\begin{figure}[H]
% \raggedright
 %\includegraphics[scale=0.04]{tu-logo}\\
%\end{figure}
\end{minipage}
\begin{minipage}{0.5\textwidth}
\begin{figure}[H]
\raggedleft
%\includegraphics[scale=0.04]{tu-logo}\\
\end{figure}
\end{minipage}
\end{tabular}\\
\\
  \small
      {
    Erste tet Hausaufgabe\\
    }


}
%pdflatex -interaction=nonstopmode
\begin{document}
\maketitle
%\author
% \begin{section*}{Aufgabe a)}%Aufgabe 1
\begin{itemize}
\item[a)] 
  
Verwenden Sie das Gauß’sche Gesetz der Elektrostatik, um das elektrische Feld innerhalb und außerhalb einer kugelförmigen Raumladung mit dem Radius $a$ und der Raumladungsdichte
\[\varrho(r)=\varrho_0 \frac{r}{a}\]
zu berechnen. Geben Sie auch die Gesamtladung der Kugel an.

Wir wissen, dass die Kugel hat eine Kugelsymmetrische Ladungsträgerdichte, wegen Symmetrie können wir unsere Integrale einfacher lösen.
Wir verwenden die integrale Gauß’sche Gesetz. 
\[\oiint_{\partial K} \vec{D} \* d \vec{A} = \iiint_{K} \varrho dV\quad | \vec{D}=\varepsilon \vec{E}\]
Wir nehmen an, dass die Material eine linear isotrop homogen Permitivität hat.
Da unsere Aufbau symmetrisch ist, folgt daraus, dass die Elektrische Flussdichte auch symmetrisch ist, d.h es hängt nur von $r$ ab, bzw $\vec{D}=\vec{D}(r)=\varepsilon \vec{E}(r)$
      \begin{align}
        \oiint_{\partial K} \varepsilon\vec{E}(r) \* d \vec{A} &= \iiint_{K} \varrho dV \quad| \frac{1}{\varepsilon} \nonumber\\
        \oiint_{\partial K} \vec{E}(r) \* d \vec{A} &= \iiint_{K} \frac{\varrho}{\varepsilon} dV \quad| d \vec{A} = r^2 \sin(\vartheta) d\vartheta d\varphi \vec{e}_r \nonumber\\
\int_0^{2\pi} \int_0^\pi \vec{E}(r) \* r^2 \sin(\vartheta) d\vartheta d\varphi \vec{e}_r &= \iiint_{K} \frac{\varrho}{\varepsilon} dV \quad| dV= r^2\sin(\vartheta)dr d\theta d\varphi \nonumber\\
\int_0^{2\pi} \int_0^\pi E_r(r)  r^2 \sin(\vartheta) d\vartheta d\varphi  &=\int_0^{2\pi} \int_0^\pi \int_0^r \frac{\varrho(\tilde{r})}{\varepsilon} \tilde{r}^2\sin(\vartheta)d\tilde{r} d\vartheta d\varphi \label{eq:1}\\
\cancel{2\pi} E_r(r)  r^2 \cancel{\int_0^\pi \sin(\vartheta) d\vartheta}  &=\cancel{2\pi} \cancel{\int_0^\pi \sin(\vartheta) d\vartheta} \int_0^r \frac{\varrho(\tilde{r})}{\varepsilon} \tilde{r}^2d\tilde{r} \nonumber\\
 E_r(r)  r^2   &=\int_0^r \frac{\varrho(\tilde{r})}{\varepsilon} \tilde{r}^2d\tilde{r} \nonumber
    \end{align}

Die Ladungsträgerdichte ist $0$ wenn $r>a$:
\[\varrho(r)=\begin{cases} 
      0 & r>a \\
      \varrho_0 \frac{r}{a} & r\leq a\\
   \end{cases}\]
%\[ E_r(r)     =\frac{1}{r^2 \varepsilon} \int_0^{\min(r,a)} \varrho(\tilde{r}) \tilde{r}^2d\tilde{r} \]
\[ E_r(r)     =\frac{1}{r^2 \varepsilon} \int_0^{\min(r,a)} \varrho_0 \frac{\tilde{r}}{a} \tilde{r}^2d\tilde{r} \]
\[ E_r(r)     =\frac{1}{r^2 \varepsilon} \int_0^{\min(r,a)} \varrho_0 \frac{1}{a} \tilde{r}^3 d\tilde{r} \]
\[ E_r(r)     =\frac{\varrho_0}{r^2 \varepsilon a} \int_0^{\min(r,a)} \tilde{r}^3 d\tilde{r} \]
\[ E_r(r)     =\frac{\varrho_0}{4 r^2 \varepsilon a} {\min(r,a)}^4 \]
Für die Gesamtladung, ersetzen wir $E_r$ in \eqref{eq:1} wenn $r=a$:
\begin{gather*}
    \int_0^{2\pi} \int_0^\pi \varepsilon E_r(r)  r^2 \sin(\vartheta) d\vartheta d\varphi = 2\pi \int_0^\pi \sin(\vartheta) d\vartheta \varepsilon E_r(r)  r^2\\
=2\pi \left(-\cos(\pi)+\cos(0)\right) \varepsilon  E_r(r)  r^2 \quad | r=a\\
=2\pi \left(-\cos(\pi)+1\right) \varepsilon  E_r(a)  a^2 = 2\pi \left(1+1\right) \varepsilon  \frac{\varrho_0}{4 a^2 \varepsilon a} a^4   a^2\\
= 4\pi \frac{\varrho_0}{4} a^3=\pi\varrho_0 a^3
\end{gather*}


%\end{section*}
%\begin{section*}{Aufgabe b)}%Aufgabe 1
    \item[b)]
     Bestimmen Sie das Potential der Raumladung innerhalb und außerhalb der Raumladung.
     
     Man kann das Potential so schreiben:
     \[\vec{E}=- \nabla \* \phi=\frac{\partial \phi}{\partial r}\vec{e}_r+\frac{1}{r}\frac{\partial \phi}{\partial \vartheta}\vec{e}_\vartheta+\frac{1}{r \sin{\vartheta}}\frac{\partial \phi}{\partial \varphi}\vec{e}_\varphi\]
     Dank kugelsymmetrie folgt:
     \[\frac{\partial \phi}{\vartheta}=0,\quad\frac{\partial \phi}{\varphi}=0\]
Somit:
\[\vec{E}=\frac{\partial \phi}{\partial r}\vec{e}_r \> \phi(r) = -\int E_r d r +\tilde{c}\]
\[\phi(r)=\int -E_r(r) dr+\tilde{c}=\int -\frac{\varrho_0}{4 r^2 \varepsilon a} {\min(r,a)}^4  dr  +\tilde{c}\]
Wir setzen die Bezugspotential im unendliche gleich $0$. \(\phi(\infty)\overset{!}{=}0\)
\[\lim_{r\rightarrow\infty}\phi(r)=-\int_0^\infty \frac{\varrho_0}{4 r^2 \varepsilon a} {\min(r,a)}^4  dr  +\tilde{c}
=-\int_0^a \frac{\varrho_0}{4 r^2 \varepsilon a} {\min(r,a)}^4  dr -\int_a^\infty \frac{\varrho_0}{4 r^2 \varepsilon a} {\min(r,a)}^4  dr  +\tilde{c} \]
\[=-\int_0^a \frac{\varrho_0}{4 r^2 \varepsilon a} {r}^4  dr -\int_a^\infty \frac{\varrho_0}{4 r^2 \varepsilon a} {a}^4  dr  +\tilde{c} \overset{!}{=}0\]
Wir evaluieren die erste Integral.
\[\int_0^a \frac{\varrho_0}{4 r^2 \varepsilon a} {r}^4  dr=\frac{\varrho_0}{4 \varepsilon a}\int_0^a \frac{r^4}{r^2} dr=\frac{\varrho_0}{4 \varepsilon a} \frac{a^3}{3}=\frac{\varrho_0 a^2}{12 \varepsilon }\]
Wir evaluieren die zweite Integral.
\[\int_a^\infty \frac{\varrho_0}{4 r^2 \varepsilon a} {a}^4  dr=\frac{a^4 \varrho_0}{4\varepsilon a}\int_a^\infty \frac{1}{r^2} dr
=\frac{a^4 \varrho_0}{4\varepsilon a}\left[-\frac{1}{r}\right]_{r=a}^{r\rightarrow\infty}
=\frac{a^3 \varrho_0}{4\varepsilon}\left(\frac{1}{a}\right)
=\frac{a^2 \varrho_0}{4\varepsilon}\]
\[0\overset{!}{=}-\frac{\varrho_0 a^2}{12 \varepsilon }-\frac{a^2 \varrho_0}{4\varepsilon}+\tilde{c}\implies\tilde{c}=\frac{a^2 \varrho_0}{3\varepsilon}\] 
Daraus folgt:
\[\phi(r)=-\int_0^r \frac{\varrho_0}{4 \tilde{r}^2 \varepsilon a} {\min(\tilde{r},a)}^4  d\tilde{r}  +\frac{a^2 \varrho_0}{3\varepsilon}\]
Es läss sich auch so darstellen:
\[\phi(r)=-\left(\int_0^{\min(a,r)} \frac{\varrho_0}{4 r^2 \varepsilon a} {r}^4  dr+\int_a^{\max(a,r)} \frac{\varrho_0}{4 r^2 \varepsilon a} {a}^4 dr\right)  +\frac{a^2 \varrho_0}{3\varepsilon}\]
\[=-\left(\frac{\varrho_0}{4 \varepsilon a} \frac{{\min(a,r)}^3}{3}+
\frac{a^3 \varrho_0}{4\varepsilon}\left(-\frac{1}{\max(a,r)}+\frac{1}{a}\right)-\frac{a^2 \varrho_0}{3\varepsilon}\right)\]
\[=-\left(\frac{\varrho_0}{\varepsilon}\left(\frac{1}{4 a} \frac{{\min(a,r)}^3}{3}+
\frac{a^3}{4}\left(-\frac{1}{\max(a,r)}+\frac{1}{a}\right)-\frac{a^2}{3}\right)\right)\]
\[=-\left(\frac{\varrho_0}{4 \varepsilon}\left(\frac{1}{a} \frac{{\min(a,r)}^3}{3}+
a^3 \left(-\frac{1}{\max(a,r)}+\frac{1}{a}\right)-\frac{4 a^2}{3}\right)\right)\]
\[=-\left(\frac{\varrho_0}{12 \varepsilon}\left(\frac{1}{a} {\min(a,r)}^3+
3a^3 \left(-\frac{1}{\max(a,r)}+\frac{1}{a}\right)-4 a^2\right)\right)\]
Um deutlicher Formeln zu bekommen können wir Fallunterschied machen.
\begin{itemize}
\item[Fall $r\leq a$]
\(\min(r,a)=r \land \max(r,a)=a\)
\[\phi(r)=-\frac{\varrho_0}{12 \varepsilon}\left(\frac{1}{a} {r}^3+
3a^3 \left(-\frac{1}{a}+\frac{1}{a}\right)-4 a^2\right)\]
\[=-\frac{\varrho_0}{12 \varepsilon}\left(\frac{1}{a} {r}^3-4 a^2\right)\]
\item[Fall $r>a$]
\(\min(r,a)=a \land \max(r,a)=r\)
\[\phi(r)=-\frac{\varrho_0}{12 \varepsilon}\left(\frac{1}{a} {a}^3+
3a^3 \left(-\frac{1}{r}+\frac{1}{a}\right)-4 a^2\right)\]
\[\phi(r)=-\frac{\varrho_0 a^2}{12 \varepsilon}\left(1+
3a \left(-\frac{1}{r}+\frac{1}{a}\right)-4 \right)\]
\[\phi(r)=-\frac{\varrho_0 a^2}{12 \varepsilon}\left(
3a \left(\frac{r-a}{a r}\right)-3 \right)\]
\[\phi(r)=-\frac{\varrho_0 a^2}{4 \varepsilon}\left(
\left(\frac{r-a}{r}\right)-1 \right)\]
\[\phi(r)=-\frac{\varrho_0 a^2}{4 \varepsilon}\left(
\left(1-\frac{a}{r}\right)-1 \right)\]
\[\phi(r)=\frac{\varrho_0 a^2}{4 \varepsilon}\left(
\frac{a}{r} \right)\]
\[\phi(r)=\frac{\varrho_0 a^3}{4 r \varepsilon}\]

\end{itemize}

 %   \end{section*}
%\begin{section*}{Aufgabe c)}%Aufgabe 1
    \item[c)]
     Wie groß ist die Kraft $\vec{F}$ auf eine Punktladung $Q$ am Ort $(0, a, 0)$, wenn sich die Mittelpunkte
zweier ungleichnamiger Raumladungen mit den Dichten $\pm\varrho_0\frac{r}{a}$ gemäß der folgenden Skizze an
den Orten $(\mp b, 0, 0)$ befinden?
  %  \end{section*}
\end{itemize}
\end{document}
