%--------------------------------------------------------%
%-------------------- 0-6_Anhang.tex --------------------%
%--------------------------------------------------------%
%----- Hier könnt ihr weiterführende Informationen, -----%
%----- umfangreiche Codelisten, Messdatensammlungen -----%
%----- oder andere Quellmaterialien einfügen. -----------%
%--------------------------------------------------------%
%--------------------------------------------------------%
%----- Achtung: Fügt bitte ausschließlich Inhalte -------%
%-----          ein, die wirklich relevant für eure -----%
%-----          Ausführungen sind! ----------------------%
%--------------------------------------------------------%
%--------------------------------------------------------%
%
\section{Verwendeter Matlab-Code}
\label{app:Matlab}
%
\begin{minted}[mathescape,linenos,   numbersep=3pt, gobble=0,  frame=lines, framesep=2mm,breaklines]{matlab}
messwerte=load("ELNW_PR_03_Vorgabe_Messwerte_20V.mat")
messwerte.I=(messwerte.u_Rpp)/57
messwerte.Z=(messwerte.u_Epp)./messwerte.I
t=readLTspice("labor3_20.txt","Nyquist")
figure(1)
clf
subplot(2,1,1);
semilogx(t(:,1)/1000 ,t(:,2)*1000)
hold on
title("Strom")
xlabel("f in kHz")
ylabel("|I| in mA")

loglog(messwerte.f/1000,messwerte.u_Rpp/57*1000,"*")
legend("Simulierte","Gemessene")
grid on
subplot(2,1,2);
loglog(t(:,1)/1000,t(:,4))
hold on
title("Impedanz")
xlabel("f in kHz")
ylabel("|Z| in \Omega")
loglog(messwerte.f/1000,messwerte.Z,"*")
grid on
figure(2)
clf
subplot(2,1,1);

semilogx(t(:,1)/1000 ,t(:,2)*1000)
hold on
title("Strom")
xlabel("f in kHz")
ylabel("|I| in mA")

semilogx(messwerte.f/1000,messwerte.u_Rpp/57*1000,"*")
legend("Simulierte","Gemessene")
grid on
subplot(2,1,2);
semilogx(t(:,1)/1000,t(:,4)/1000)
hold on
title("Impedanz")
xlabel("f in kHz")
ylabel("|Z| in k\Omega")
semilogx(messwerte.f/1000,messwerte.Z/1000,"*")
grid on
\end{minted}

\section{Verwendeter Python-Code}
\label{app:Python}
\inputminted{python}{src/test.py}
%
%
%
% Auch im Anhang sollte die Autorenschaft gekennzeichnet werden. Ihr könnt aber im Falle von Scilab-Code die Autorenschaft im Code selbst kennzeichnen.
%
\begin{flushright}
  \textit{\autorA}
\end{flushright}
%
%
%
%\newpage
%\section{Datenblätter}
%\label{app:Datenblätter}
%
%\lipsum[7-9]
%
%
%
% Auch wenn es sich nur um Abbildungssammlungen handelt, muss die Autorenschaft trotzdem gekennzeichnet werden.
%
%\begin{flushright}
  %\textit{\autorA}
%\end{flushright}
%
%
%