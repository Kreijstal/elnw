%--------------------------------------%
%----- 3_Versuchsdurchführung.tex -----%
%--------------------------------------%
%--------------------------------------%
%
\subsection{Messung}
\label{subsec:3_Messung}
%
%------------------------------%
%----- Beginn eures Teils -----%
%------------------------------%
%
\begin{figure}[H]
  \Large
  \centering
  \begin{tikzpicture}[circuit ee IEC, font = \sffamily]
    \matrix(M)[
      matrix of nodes, nodes in empty cells,
      inner sep = 0pt, outer sep = -.1\pgflinewidth,
      column sep = 20mm, row sep = 15mm,
      nodes = {minimum width = 0pt}
      ]
    {
      & & & & & \\
      & & & & & \\
      & & & & & \\
      & & & & & \\
      & & & & & \\
    };
    %
    \draw[circuit symbol unit = 15pt]
      (M-2-2) to [voltage source] (M-5-2);
    \draw[circuit symbol unit = 15pt]
      (M-2-2) to [resistor = {info = $R$, info' = \SI{56}{\ohm}}] (M-2-4);
    \draw[circuit symbol unit = 15pt]
      (M-2-4) to [capacitor = {info = $C$, info' = \SI{10}{\nano\farad}}] (M-2-6);
    \draw[circuit symbol unit = 15pt]
      (M-2-6) to [inductor = {info = $L$, info' = \SI{470}{\micro\henry}}] (M-5-6);
    %
    \draw (M-5-2)--(M-5-6);
    %
    \draw[UPfeil = 6mm]
      ([xshift = -9mm]M-2-2.east)--([xshift = -9mm]M-5-2.east)
	  node[midway, left]{$\underline{U}_\mathrm{E}$};
    \draw[UPfeil = 16mm]
      ([yshift = 13mm]M-2-2.south)--([yshift = 13mm]M-2-4.south)
      node[midway, above]{$\underline{U}_\mathrm{R}$};
    \draw[UPfeil = 16mm]
      ([yshift = 13mm]M-2-4.south)--([yshift = 13mm]M-2-6.south)
      node[midway, above]{$\underline{U}_\mathrm{C}$};

    \draw[UPfeil = 6mm]
      ([xshift = 14mm]M-2-6.west)--([xshift = 14mm]M-5-6.west)
      node[midway, right]{$\underline{U}_\mathrm{L}$};
    \draw[IPfeil = 4mm]
      ([xshift = 6mm]M-2-3.center)--([xshift = 6mm]M-2-4.center)
      node[midway, above]{$\underline{I}$};
    %
  \end{tikzpicture}
  \caption{Schaltbild der untersuchten RLC-Reihenschaltung}
  \label{fig:3_Schaltung}
\end{figure}


%
%
%
\begin{flushright}
  \textit{\autorA}
\end{flushright}
%
%------------------------------%
%------ Ende eures Teils ------%
%------------------------------%
%
%
%
\subsection{Simulation}
\label{subsec:3_Simulation}
%
%------------------------------%
%----- Beginn eures Teils -----%
%------------------------------%
%

%
%
%
%\begin{flushright}
  %\textit{\autorA}
%\end{flushright}
%
%------------------------------%
%------ Ende eures Teils ------%
%------------------------------%
%
%
%
