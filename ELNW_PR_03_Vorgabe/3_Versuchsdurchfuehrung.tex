%--------------------------------------%
%----- 3_Versuchsdurchführung.tex -----%
%--------------------------------------%
%--------------------------------------%
%
\subsection{Messung}
\label{subsec:3_Messung}
%
%------------------------------%
%----- Beginn eures Teils -----%
%------------------------------%
%
\begin{figure}[H]
  \Large
  \centering
  \begin{tikzpicture}[circuit ee IEC, font = \sffamily]
    \matrix(M)[
      matrix of nodes, nodes in empty cells,
      inner sep = 0pt, outer sep = -.1\pgflinewidth,
      column sep = 20mm, row sep = 15mm,
      nodes = {minimum width = 0pt}
      ]
    {
      & & & & & \\
      & & & & & \\
      & & & & & \\
      & & & & & \\
      & & & & & \\
    };
    %
    \draw[circuit symbol unit = 15pt]
      (M-2-2) to [voltage source] (M-5-2);
    \draw[circuit symbol unit = 15pt]
      (M-2-2) to [resistor = {info = $R$, info' = \SI{56}{\ohm}}] (M-2-4);
    \draw[circuit symbol unit = 15pt]
      (M-2-4) to [capacitor = {info = $C$, info' = \SI{10}{\nano\farad}}] (M-2-6);
    \draw[circuit symbol unit = 15pt]
      (M-2-6) to [inductor = {info = $L$, info' = \SI{470}{\micro\henry}}] (M-5-6);
    %
    \draw (M-5-2)--(M-5-6);
    %
    \draw[UPfeil = 6mm]
      ([xshift = -9mm]M-2-2.east)--([xshift = -9mm]M-5-2.east)
	  node[midway, left]{$\underline{U}_\mathrm{E}$};
    \draw[UPfeil = 16mm]
      ([yshift = 13mm]M-2-2.south)--([yshift = 13mm]M-2-4.south)
      node[midway, above]{$\underline{U}_\mathrm{R}$};
    \draw[UPfeil = 16mm]
      ([yshift = 13mm]M-2-4.south)--([yshift = 13mm]M-2-6.south)
      node[midway, above]{$\underline{U}_\mathrm{C}$};

    \draw[UPfeil = 6mm]
      ([xshift = 14mm]M-2-6.west)--([xshift = 14mm]M-5-6.west)
      node[midway, right]{$\underline{U}_\mathrm{L}$};
    \draw[IPfeil = 4mm]
      ([xshift = 6mm]M-2-3.center)--([xshift = 6mm]M-2-4.center)
      node[midway, above]{$\underline{I}$};
    %
  \end{tikzpicture}
  \caption{Schaltbild der untersuchten RLC-Reihenschaltung}
  \label{fig:3_Schaltung}
\end{figure}

Wir benötigen: 
\begin{itemize}
    \item einen Schichtwiderstand ($56\si{\ohm}$), 
    \item einen Kondensator ($10\si{\nano\farad}$)
    \item eine Spule ($470\si{\micro\henry}$) 
    \item ein BNC-BNC-Kabel
    \item ein Oszilloskop, 
    \item einen Funktionsgenerator, 
    \item ein Steckbrett
\end{itemize}


\par Zuerst werden wir die Spule, den Widerstand und den Kondensator in Reihe auf dem Steckbrett und unsere Spannungsquelle ($V_a$ und $V_b$) parallel zu unserer RLC-Reihenschaltung verbunden.
\par Unser Oszilloskop wird die Spannung des Widerstand messen, daher sollten $V_c$ und Erde parallel zum Widerstand geschaltet werden. Dann verbinden wir unseren Funktionsgenerator einen BNC-BNC-Kabel mit dem Oszilloskop. Der Generator sollte auch mit einem BNC-Doppelbananenkabel, das in $V_a$ und $V_b$ angeschlossen ist, an das Steckbrett angeschlossen werden.
\par Das Oszilloskop muss unseren Widerstandspannung messen, daher ist es über $V_c$ und Masse mit einem BNC-Doppelbananenkabel verbunden. Im Funktionsgenerator wird eine Sinusfunktion mit einer Spitze-Spitze-Spannung von 10V eingestellt. Wir beginnen mit einer Frequenz von 1 kHz.
Wir verwenden die Messfunktion unseres Oszilloskops, um die pp-Spannungen von $U_E$ und $U_R$ und die Frequenz zu lesen. \par Wenn die Eingangsspannung im Oszilloskop nicht 10 V beträgt, ändern wir sie in unserem Funktionsgenerator so, dass das Oszilloskop knapp 10 V anzeigt. Gemäß unserem Testblatt notieren wir unsere Spannungen für die folgenden Frequenzen: 1 bis 10 kHz (schritt:1 kHz), 10 bis 65 kHz (schritt:5 kHz), 65, 80 kHz (schritt:1 kHz), 80 bis 110 kHz (schritt:5 kHz),  110 bis 200 kHz (schritt:10 kHz), 250 bis 500 kHz (schritt:50 kHz), 500 bis 1000 kHz (schritt:100 kHz).  
\par Unsere Resonanzfrequenz sollte bei 73 kHz liegen. Wir sehen, dass die Spannungen in Phase sind. Bei etwa 1 MHz ist $U_R$ viel kleiner und weist eine Phasendifferenz zu $U_E$ auf. Dies geschieht, weil die Spule bei höheren Frequenzen dominiert.
%
Wir können die Komponenten als Impedanzen betrachten, die Reihenfolge der Impedanzen spielen keine Rolle.
%
%
\begin{flushright}
  \textit{\autorA}
\end{flushright}
%
%------------------------------%
%------ Ende eures Teils ------%
%------------------------------%
%
%
%
\subsection{Simulation}
\label{subsec:3_Simulation}

%------------------------------%
%----- Beginn eures Teils -----%
%------------------------------%
%
Wir benötigen eine Spannungsquelle mit einer Amplitude von 10V und eine Phasenverschiebung von 0 rad, die geeignet für Wechselstromanalyse ist, (ihre Wert ist \lstinline{AC 10 0}),  Wir brauchen einen eine $470\si{\micro\henry}$ Spule, $56\si{\ohm}$ Widerstand, draht, und einen 10nF Kondensator. Diese werden in Reihe verbunden. Um Spannung zum plotten wir müssen ein vorgegeben Potential bzw GND/Masse abgeben, wir definieren, dass es muss gleich wie die Potential von den negativen Pol der Quelle sein.

Wir wollen auch eine AC analysis Kommando hinzufügen, so:  \lstinline{.ac dec 100 1k 1Meg}
Führen wir aus, fügen wir Traces hinzu, Wir wählen eine Logarithmische Darstellung aus.
%
%
%
%\begin{flushright}
  %\textit{\autorA}
%\end{flushright}
%
%------------------------------%
%------ Ende eures Teils ------%
%------------------------------%
%
%
%
