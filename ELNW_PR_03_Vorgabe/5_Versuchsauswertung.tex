%------------------------------------%
%----- 5_Versuchsauswertung.tex -----%
%------------------------------------%
%------------------------------------%
%
%------------------------------%
%----- Beginn eures Teils -----%
%------------------------------%
%
Wir können sehen, dass die gemessene Strom höher als die simulierte Impedanz ist, außer in der Nähe von der Resonanzfrequenz wo die Theoretische werte deutlich höher sind.

In Bezug auf die Impedanz, die gemessene Impedanz ist immer großer als die Theoretische Impedanz außer in der Nähe von der Resonanzfrequenz wo die Theoretische werte kleiner sind.

Bei 1000 Hz, die gemessene Strome fängt mit 1,89 mA an , wobei der theoretische Werte 0,628 mA beträgt. Die theoretische und gemessene Impedanz betragen 15912,64 bzw. 10555,5 Ω  

Bei Resonanzfrequenz (73 413 Hz) findet sich maximal Strom und minimal Impedanz.

Der maximale gemessene Strom macht 284,8 mA aus, wobei für die Theoretische werte ungefähr 350 mA ist. Die theoretische und gemessene Impedanz Minimum betragen ungefähr 57 bzw.  70 Ω  

Bei 1 MHz, der theoretische und gemessene Strom betragen 3,4 bzw. 3,298 mA, und die theoretische und gemessene Impedanz entspricht 2937,7 bzw. 3031,9 Ω

Im Versuchsergebnis sahen wir eine RLC-Reihenschschaltung. Die Schwingendesysteme verursachen aufgrund des Kondensators und der Spüle (als elektrischer Energiespeicher bzw magnetische Energie) eine Resonanzfrequenz. Mit der Resonanzfrequenz kann das Minimum die Gesamtimpedanz und damit den höchsten Strom erreichen.

Die absolute Impedanz sinkt ein Vielfaches der Große von anderen Frequenzen, wenn wir uns der Resonanzfrequenz nähern, wobei mit verringerter Impedanz mehr Strom fließt. Aber natürlich gibt es Abweichungen, weil es in Wirklichkeit keine idealen Bauelemente gibt, ein Teil der Energie durch Wärme verloren geht usw.
%
%
%
%\begin{flushright}
  %\textit{\autorA}
%\end{flushright}
%
%------------------------------%
%------ Ende eures Teils ------%
%------------------------------%
%
%
%