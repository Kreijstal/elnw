%---------------------------------------------------------%
%-------------------- 0-5_Quellen.tex --------------------%
%---------------------------------------------------------%
%----- Hier sollt ihr eure Quellen angeben. --------------%
%---------------------------------------------------------%
%---------------------------------------------------------%
%----- Achtung: Direkte und indirekte Zitate müssen ------%
%-----          in jedem Fall gekennzeichnet werden. -----%
%-----          Verstöße werden als Betrugsversuch -------%
%-----          und entsprechend mit 0 PP bewertet! ------%
%-----          Außerdem kann ein Plagiatsvorwurf zu -----%
%-----          weiteren Untersuchungen führen! ----------%
%---------------------------------------------------------%
%---------------------------------------------------------%
%
\begin{sloppy}
\begin{small}
\begin{onehalfspacing}
  \begin{thebibliography}{--------------------} % Bestimmt den Platz, den Bezeichner erhalten.
  %
  % Am besten alphabetisch nach dem Bezeichner ('[Xyz]') sortieren!
  %
  %
  %
  \bibitem[Laborskript]{src:Skript}
    Teske, P., Gornig, C.: "`Einführung in das Praktikum Elektrische Netzwerke"', Skript zum 0. Versuch im Laborpraktikum des Moduls Elektrische Netzwerke, SoSe 2018, TU Berlin\\
    \url{https://isis.tu-berlin.de/pluginfile.php/900997/mod_resource/content/1/PR00_Einf\%C3\%BChrung.pdf}\\
    Stand: 19.04.2018
%
% Vorlagen:
%
  %\bibitem[]{src:}
    %: "`"', , \\
    %\url{}\\
    %Stand:
  %
  % ODER:
  %
  %\bibitem[]{src:}
    %: "`"', , \\
    %\url{}\\
    %Zugriff: , Uhr
  \end{thebibliography}
\end{onehalfspacing}
\end{small}
\end{sloppy}
