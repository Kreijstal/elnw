%--------------------------------------------------------%
%-------------------- 0-6_Anhang.tex --------------------%
%--------------------------------------------------------%
%----- Hier könnt ihr weiterführende Informationen, -----%
%----- umfangreiche Codelisten, Messdatensammlungen -----%
%----- oder andere Quellmaterialien einfügen. -----------%
%--------------------------------------------------------%
%--------------------------------------------------------%
%----- Achtung: Fügt bitte ausschließlich Inhalte -------%
%-----          ein, die wirklich relevant für eure -----%
%-----          Ausführungen sind! ----------------------%
%--------------------------------------------------------%
%--------------------------------------------------------%
%
\section{Verwendeter Scilab-Code}
\label{app:Scilab}
%
\lipsum[5-6]
%
%
%
% Auch im Anhang sollte die Autorenschaft gekennzeichnet werden. Ihr könnt aber im Falle von Scilab-Code die Autorenschaft im Code selbst kennzeichnen.
%
\begin{flushright}
  \textit{\autorA}
\end{flushright}
%
%
%
\newpage
\section{Datenblätter}
\label{app:Datenblätter}
%
\lipsum[7-9]
%
%
%
% Auch wenn es sich nur um Abbildungssammlungen handelt, muss die Autorenschaft trotzdem gekennzeichnet werden.
%
\begin{flushright}
  \textit{\autorA}
\end{flushright}
%
%
%