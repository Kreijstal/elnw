%--------------------------------------------------------%
%-------------------- 0-6_Anhang.tex --------------------%
%--------------------------------------------------------%
%----- Hier könnt ihr weiterführende Informationen, -----%
%----- umfangreiche Codelisten, Messdatensammlungen -----%
%----- oder andere Quellmaterialien einfügen. -----------%
%--------------------------------------------------------%
%--------------------------------------------------------%
%----- Achtung: Fügt bitte ausschließlich Inhalte -------%
%-----          ein, die wirklich relevant für eure -----%
%-----          Ausführungen sind! ----------------------%
%--------------------------------------------------------%
%--------------------------------------------------------%
%
\section{Verwendeter Matlab-Code}
\label{app:Matlab}
%
\begin{minted}[mathescape,
               linenos,
               numbersep=3pt,
               gobble=0,
               frame=lines,
               framesep=2mm,breaklines]{matlab}
figure(1)
clf
hold on
plot(t1.data(:,1) ,t1.data(:,3))
plot(t1.data(:,1) ,t1.data(:,2))

%nonsense plot(t1.data(:,1) ,t1.data(:,2))
title("Eingangs und Ausgangstrome - U_2")
xlabel("t in s")
ylabel("U in V")
yyaxis right
ylabel("I in A")
plot(t1.data(:,1) ,t1.data(:,6))
plot(t1.data(:,1) ,t1.data(:,10))
legend("U_2" ,"U_1","I_{ein}","I_{aus}")
grid on
figure(2)
clf
hold on
plot(t2(:,1) ,t2(:,2))
plot(t2(:,1) ,t2(:,3))

%nonsense plot(t1.data(:,1) ,t1.data(:,2))
title("Eingangs und Ausgangstrome - U_1")
xlabel("t in s")
ylabel("U in V")
yyaxis right
ylabel("I in A")
plot(t2(:,1) ,t2(:,10))
plot(t2(:,1) ,t2(:,6))
legend("U_1","U_2","I_{ein}","I_{aus}")
grid on
\end{minted}

%\section{Verwendeter Python-Code}
%\label{app:Python}
%\inputminted{python}{src/reader.py}
%
%
%
% Auch im Anhang sollte die Autorenschaft gekennzeichnet werden. Ihr könnt aber im Falle von Scilab-Code die Autorenschaft im Code selbst kennzeichnen.
%
\begin{flushright}
  \textit{\autorA}
\end{flushright}
%
%
%
%\newpage
%\section{Datenblätter}
%\label{app:Datenblätter}
%
%\lipsum[7-9]
%
%
%
% Auch wenn es sich nur um Abbildungssammlungen handelt, muss die Autorenschaft trotzdem gekennzeichnet werden.
%
%\begin{flushright}
  %\textit{\autorA}
%\end{flushright}
%
%
%