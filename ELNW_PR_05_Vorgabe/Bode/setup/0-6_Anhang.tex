%--------------------------------------------------------%
%-------------------- 0-6_Anhang.tex --------------------%
%--------------------------------------------------------%
%----- Hier könnt ihr weiterführende Informationen, -----%
%----- umfangreiche Codelisten, Messdatensammlungen -----%
%----- oder andere Quellmaterialien einfügen. -----------%
%--------------------------------------------------------%
%--------------------------------------------------------%
%----- Achtung: Fügt bitte ausschließlich Inhalte -------%
%-----          ein, die wirklich relevant für eure -----%
%-----          Ausführungen sind! ----------------------%
%--------------------------------------------------------%
%--------------------------------------------------------%
%
\section{Verwendeter Matlab-Code}
\label{app:Matlab}
%
\begin{minted}[mathescape,
               linenos,
               numbersep=3pt,
               gobble=0,
               frame=lines,
               framesep=2mm,breaklines]{matlab}
messwerte=load("ELNW_PR05_Vorgabe_Messwerte.mat")
messwerte.Amplitude=20.*log(messwerte.U_App./messwerte.U_Epp)./log(10)
messwerte.phi=360.*messwerte.dt.*messwerte.f
m = readLTspice('labor5_1.txt','bode');
m2 = readLTspice('labor5_2.txt','bode');
figure(1)   %Amplitude
clf
semilogx(m(:,1), m(:,2),'b')    %Schaltung 1
hold on
semilogx(m2(:,1), m2(:,2),'r')
semilogx(messwerte.f, messwerte.Amplitude,'g')
%propertyeditor('on')
set(gca,'xscale','log')
ylabel("|H|dB")
yyaxis right
semilogx(0, 0,'--k')
semilogx(m(:,1), m(:,3),'--b')
semilogx(m2(:,1), m2(:,3),'--r')
semilogx(messwerte.f, messwerte.phi,'--g')
%plot(m(:,1), m(:,3),'r')    
%plot(m(:,1),20*log10(U_App/U_Epp),'g')
ylabel({"$\varphi_{\underline H}$ in ${}^{\circ}$"},"interpreter","latex")
xlabel("Frequenz in Hz")
title(['Amplitude in dB/Phase in ' char(176)])
[legh,objh]=legend("Sim R=1M","Sim R=100M","Gemessene","Phase")
t=(findobj(objh,'type','line'))
t(7).Color=[0,0,0]
grid on
hold off
\end{minted}

\section{Verwendeter Python-Code}
\label{app:Python}
\inputminted{python}{src/test.py}
%
%
%
% Auch im Anhang sollte die Autorenschaft gekennzeichnet werden. Ihr könnt aber im Falle von Scilab-Code die Autorenschaft im Code selbst kennzeichnen.
%
\begin{flushright}
  \textit{\autorA}
\end{flushright}
%
%
%
%\newpage
%\section{Datenblätter}
%\label{app:Datenblätter}
%
%\lipsum[7-9]
%
%
%
% Auch wenn es sich nur um Abbildungssammlungen handelt, muss die Autorenschaft trotzdem gekennzeichnet werden.
%
%\begin{flushright}
  %\textit{\autorA}
%\end{flushright}
%
%
%