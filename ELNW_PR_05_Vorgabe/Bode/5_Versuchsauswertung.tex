%------------------------------------%
%----- 5_Versuchsauswertung.tex -----%
%------------------------------------%
%------------------------------------%
%
%------------------------------%
%----- Beginn eures Teils -----%
%------------------------------%
%
\subsubsection{Phase}
Zu Beginn verhalten sich die Messung und die Schaltung 1 insofern gleich, als sie mit zunehmender Frequenz abnehmen. Im Gegensatz dazu nimmt die Phase zu. In der Nähe der Frequenz = 100 Hz steigen alle Phasen an. Ab Frequenz = 1000 Hz nehmen alle Phasen ab und die Differenz zwischen ihnen wird immer kleiner.

\subsubsection{Amplitude}
Bei einer Frequenz von 10 Hz haben die Schaltung mit dem größten Messwiderstand (RMess), Schaltung 2 und die Schaltung der Messung eine sehr ähnliche Amplitude. Im Gegensatz dazu ist die Amplitude der Schaltung mit der kleinsten RMess, Schaltung 1, höher. Die Amplitude der beiden Schaltkreise und die Messung nehmen zu, je höher die Frequenz ist. Im Frequenzintervall $\qty(102,\dots,103)$ ist die Differenz zwischen den drei Amplituden gering. Die Amplituden beider simulierter Schaltungen sind dann ähnlich und größer als die Amplitude der Messung. Die Amplitude ist direkt proportional zur Ausgangsspannung und umgekehrt proportional zur Eingangsspannung. Die Eingangsspannung der Messung ist doppelt so hoch wie die Spannung beider simulierter Schaltkreise. Da jedoch die RMess der Messung viel kleiner als die RMess der Schaltung 2 ist, sind ihre Amplituden nahezu gleich.

\subsubsection{Ähnlichkeiten}
Es ist sehr Ähnlich von Gestalt wie die Simulation wenn $R=1 \si{\mega\ohm}$, aber die gemessene Werte ist mehr chaotisch und nicht so glatt. 


\subsubsection{Unterschiede}
Die mögliche Ursache für die Unterschiede ist RMess, da dies bedeutet, dass die Ausgangsspannung höher oder niedriger sein kann. Auch hier spielt die Zeitverschiebung eine sehr wichtige Rolle, da die Phase direkt proportional dazu ist. Wenn 
$f_E\rightarrow\infty$ ist dann dB ist nicht 0 in den gemessene werte sondern viel kleiner (üngefähr -2 dB  )
%

%
Bei kleinen Frequenzen bildet sich ein Stromteiler zwischen $R_{Mess}$ und $R_2$, $C_2$, da die Impedanz des Kondensators sehr groß wird und ein kleiner Strom über den Messwiderstand abfließt. 
%
\begin{flushright}
  \textit{\autorA}
\end{flushright}
%
%------------------------------%
%------ Ende eures Teils ------%
%------------------------------%
%
%
%