%------------------------------------%
%----- 4_Versuchsergebnisse.tex -----%
%------------------------------------%
%------------------------------------%
%
\subsection{Messdaten}
\label{subsec:4_Daten}
%
Die Messergebnisse sind in Tab. \ref{tab:4_Messdaten} dargestellt\footnote{Alle Werte aufgenommen bzw. berechnet von \autorA}.
%
\begin{table}[H]
  \small
  \centering
	\caption{Messergebnisse}
	\label{tab:4_Messdaten}
	\begin{tabular}{rrrrrr}
	  \toprule
		%
	  \multicolumn{1}{c}{Einstellwerte} &
		\multicolumn{3}{c}{Messwerte} &
		\multicolumn{2}{c}{Berechnete Werte} \\
		\cmidrule(lr{1mm}){1-1}\cmidrule(lr{1mm}){2-4}\cmidrule(lr{1mm}){5-6}
		%
		$f_\mathrm{E}$ in \si{\hertz} &
			$U_\mathrm{E,pp}$ in \si{\volt} &
    		$U_\mathrm{A,pp}$ in \si{\volt} &
    		$\Delta t$ in \si{\micro\second} &
    			$|\underline{H}|_\mathrm{dB}$ in \si{\decibel} &
				$\mathrm{arg}(\underline{H})$ in \si{\degree}\\
		\midrule
		%
    \input{src/4_Messdaten.txt}
    %
		\bottomrule
	\end{tabular}
\end{table}
%
Die Formeln dafür sind
\[m|\underline{H}|_{dB}=20\log(\frac{U_{A,pp}}{U_{E,pp}})\]
\[{\varphi}_{\underline H}=360^\circ \cdot arg(\underline{H})=360^\circ \cdot f_E \cdot \Delta t \]
%
%
\newpage
\subsection{Ergebnisplots}
\label{subsec:4_Plots}
%
%------------------------------%
%----- Beginn eures Teils -----%
%------------------------------%
%

%
%
%
%\begin{flushright}
  %\textit{\autorA}
%\end{flushright}
%
%------------------------------%
%------ Ende eures Teils ------%
%------------------------------%
%
%
%