%------------------------------------%
%----- 4_Versuchsergebnisse.tex -----%
%------------------------------------%
%------------------------------------%
%pdflatex -interaction=nonstopmode --shell-escape 
\subsection{Messdaten: Impedanzmatrix}
\label{subsec:4_Daten}
%
%------------------------------%
%----- Beginn eures Teils -----%
%------------------------------%
%
%
Wir haben die folgenden Messungen:
\begin{table}[H]
  \small
  \centering
	\caption{Messergebnisse}
	\label{tab:Messdaten}
	\begin{tabular}{rrrrrrr}
	  \toprule
		%

    \input{Messwerte/ELNW_PR_04_Vorgabe_Messwerte.txt}

    \midrule

		\bottomrule
	\end{tabular}
\end{table}

Um die Impedanzmatrix zu berechnen können wir folgenden Formeln anwenden.
\[\underline{Z}_{11} = \frac{\underline{U}_1}{\underline{I}_{1}}\Bigr|_{\underline{I}_2=0} \]
\[\underline{Z}_{21} = \frac{\underline{U}_2}{\underline{I}_{1}}\Bigr|_{\underline{I}_2=0} \]
\[\underline{Z}_{12} = \frac{\underline{U}_1}{\underline{I}_{2}}\Bigr|_{\underline{I}_1=0} \]
 \[\underline{Z}_{22} = \frac{\underline{U}_2}{\underline{I}_{2}}\Bigr|_{\underline{I}_1=0} \]
%
und
\[\phi_{phase}=2\pi f \Delta t\]
\[ \underline{U}_1 = \frac{1}{2} U_{1,pp}\ee^{\imag \omega \Delta t_1} = \frac{1}{2} U_{1,pp}\ee^{\imag 2 \pi f \Delta t_1} \]
\[ \underline{I}_1 = \frac{\underline{U}_E-\underline{U}_1}{R_{Shunt}} =(\frac{1}{2})\frac{\underline{U}_{E,pp}-\underline{U}_{1,pp}\ee^{\imag  \omega \Delta t_1}}{R_{Shunt}} \]
\[\begin{pmatrix}
 Z_{11} & Z_{12} \\
 Z_{21} & Z_{22} \\
\end{pmatrix}=\begin{pmatrix}
    \frac{\underline{U}_1}{\underline{I}_{1}}\Bigr|_{\underline{I}_2=0}&
    \frac{\underline{U}_1}{\underline{I}_{2}}\Bigr|_{\underline{I}_1=0}\\
    \frac{\underline{U}_2}{\underline{I}_{1}}\Bigr|_{\underline{I}_2=0}&
    \frac{\underline{U}_2}{\underline{I}_{2}}\Bigr|_{\underline{I}_1=0} 
\end{pmatrix}
\]
\[\begin{pmatrix}
 Z_{11} & Z_{12} \\
 Z_{21} & Z_{22} \\
\end{pmatrix}=\begin{pmatrix}
    \frac{\frac{1}{2} U_{{1,pp}_{\mathrm{1}}}\ee^{\imag 2 \pi f \Delta t_{1_\mathrm{1}}} }{\frac{1}{2}\frac{\underline{U}_{{E,pp}_{\mathrm{1}}}-\underline{U}_{{1,pp}_{\mathrm{1}}}\ee^{\imag  \omega \Delta {t_1}_{\mathrm{1}}}}{R_{Shunt}}}&
    \frac{\frac{1}{2} U_{{1,pp}_{\mathrm{2}}}\ee^{\imag 2 \pi f \Delta t_{1_\mathrm{2}}} }{\frac{1}{2}\frac{\underline{U}_{{E,pp}_{\mathrm{2}}}-\underline{U}_{{2,pp}_{\mathrm{2}}}\ee^{\imag  \omega \Delta {t_2}_{\mathrm{2}}}}{R_{Shunt}}}\\
    \frac{\frac{1}{2} U_{{2,pp}_{\mathrm{1}}}\ee^{\imag 2 \pi f \Delta t_{2_\mathrm{1}}} }{\frac{1}{2}\frac{\underline{U}_{{E,pp}_{\mathrm{1}}}-\underline{U}_{{1,pp}_{\mathrm{1}}}\ee^{\imag  \omega \Delta {t_1}_{\mathrm{1}}}}{R_{Shunt}}}&
    \frac{\frac{1}{2} U_{{2,pp}_{\mathrm{2}}}\ee^{\imag 2 \pi f \Delta t_{2_\mathrm{2}}} }{\frac{1}{2}\frac{\underline{U}_{{E,pp}_{\mathrm{2}}}-\underline{U}_{{2,pp}_{\mathrm{2}}}\ee^{\imag  \omega \Delta {t_2}_{\mathrm{2}}}}{R_{Shunt}}} 
\end{pmatrix}
\]
\[\begin{pmatrix}
 Z_{11} & Z_{12} \\
 Z_{21} & Z_{22} \\
\end{pmatrix}=\begin{pmatrix}
    \frac{U_{{1,pp}_{\mathrm{1}}}\ee^{\imag 2 \pi f \Delta t_{1_\mathrm{1}}} }{\frac{\underline{U}_{{E,pp}_{\mathrm{1}}}-\underline{U}_{{1,pp}_{\mathrm{1}}}\ee^{\imag  \omega \Delta {t_1}_{\mathrm{1}}}}{R_{Shunt}}}&
    \frac{U_{{1,pp}_{\mathrm{2}}}\ee^{\imag 2 \pi f \Delta t_{1_\mathrm{2}}} }{\frac{\underline{U}_{{E,pp}_{\mathrm{2}}}-\underline{U}_{{2,pp}_{\mathrm{2}}}\ee^{\imag  \omega \Delta {t_2}_{\mathrm{2}}}}{R_{Shunt}}}\\
    \frac{ U_{{2,pp}_{\mathrm{1}}}\ee^{\imag 2 \pi f \Delta t_{2_\mathrm{1}}} }{\frac{\underline{U}_{{E,pp}_{\mathrm{1}}}-\underline{U}_{{1,pp}_{\mathrm{1}}}\ee^{\imag  \omega \Delta {t_1}_{\mathrm{1}}}}{R_{Shunt}}}&
    \frac{ U_{{2,pp}_{\mathrm{2}}}\ee^{\imag 2 \pi f \Delta t_{2_\mathrm{2}}} }{\frac{\underline{U}_{{E,pp}_{\mathrm{2}}}-\underline{U}_{{2,pp}_{\mathrm{2}}}\ee^{\imag  \omega \Delta {t_2}_{\mathrm{2}}}}{R_{Shunt}}} 
\end{pmatrix}
\]
\[\begin{pmatrix}
 Z_{11} & Z_{12} \\
 Z_{21} & Z_{22} \\
\end{pmatrix}=R_{Shunt}\begin{pmatrix}
    \frac{U_{{1,pp}_{\mathrm{1}}}\ee^{\imag 2 \pi f \Delta t_{1_\mathrm{1}}} }{\underline{U}_{{E,pp}_{\mathrm{1}}}-\underline{U}_{{1,pp}_{\mathrm{1}}}\ee^{\imag  \omega \Delta {t_1}_{\mathrm{1}}}}&
    \frac{U_{{1,pp}_{\mathrm{2}}}\ee^{\imag 2 \pi f \Delta t_{1_\mathrm{2}}} }{\underline{U}_{{E,pp}_{\mathrm{2}}}-\underline{U}_{{2,pp}_{\mathrm{2}}}\ee^{\imag  \omega \Delta {t_2}_{\mathrm{2}}}}\\
    \frac{ U_{{2,pp}_{\mathrm{1}}}\ee^{\imag 2 \pi f \Delta t_{2_\mathrm{1}}} }{\underline{U}_{{E,pp}_{\mathrm{1}}}-\underline{U}_{{1,pp}_{\mathrm{1}}}\ee^{\imag  \omega \Delta {t_1}_{\mathrm{1}}}}&
    \frac{ U_{{2,pp}_{\mathrm{2}}}\ee^{\imag 2 \pi f \Delta t_{2_\mathrm{2}}} }{\underline{U}_{{E,pp}_{\mathrm{2}}}-\underline{U}_{{2,pp}_{\mathrm{2}}}\ee^{\imag  \omega \Delta {t_2}_{\mathrm{2}}}} 
\end{pmatrix}
\]
\begin{sagesilent}
import pandas as pd
import scipy.io
import numpy as np
import sympy as sp
mat=scipy.io.loadmat('./Messwerte/ELNW_PR_04_Vorgabe_Messwerte.mat')

mat = {k:v for k, v in mat.items() if k[0] != '_'}
data = pd.DataFrame({k: np.array(v).flatten() for k, v in mat.items()})
R_shunt,f,U1pp1,U2pp1,U1pp2,U2pp2,dt11,dt12,dt21,dt22 = var("R_shunt,f,U1pp1,U2pp1,U1pp2,U2pp2,dt11,dt12,dt21,dt22")
R_shunt=1000
f=20000
dt11=data.Z11_dt_U1.squeeze()
dt12=data.Z12_dt_U1.squeeze()
dt21=data.Z21_dt_U2.squeeze()
dt22=data.Z22_dt_U2.squeeze()
U1pp1=data.Z11_U_1pp.squeeze()*exp(1j*2*pi*f*dt11)
U2pp1=data.Z21_U_2pp.squeeze()*exp(1j*2*pi*f*dt21)
U2pp2=data.Z22_U_2pp.squeeze()*exp(1j*2*pi*f*dt22)
U1pp2=data.Z12_U_1pp.squeeze()*exp(1j*2*pi*f*dt12)
Z11=U1pp1/(data.Z11_U_Epp.squeeze()-U1pp1)
Z21=U2pp1/(data.Z21_U_Epp.squeeze()-U1pp1)
Z12=U1pp2/(data.Z12_U_Epp.squeeze()-U2pp2)
Z22=U2pp2/(data.Z22_U_Epp.squeeze()-U2pp2)
p=lambda _:sp.latex(sp.sympify(_),decimal_separator="comma")
#roundc=lambda c,n:complex(round(c.real(),n),round(c.imag(),n))
\end{sagesilent}
\[Z=\sage{n(R_shunt*matrix([[Z11,Z12],[Z21,Z22]]))}\]

Da wir wissen, dass unsere Schaltung eine T-Netzwerke ist, können wir versuchen die Impedanzen der T komponenten von der Matrize züruckgewinnen.
   
    \[\underline{Z}_1=\underline{Z}_{11}-\underline{Z}_{21}\]
    \[\underline{Z}_2=\underline{Z}_{22}-\underline{Z}_{21}\]
    \[\underline{Z}_3=\underline{Z}_{12}=\underline{Z}_{21}\]
    Da wir wissen, dass unsere Schaltung eine T-Netzwerke ist, können wir versuchen die Impedanzen der T komponenten von der Matrize züruckgewinnen.
    \begin{figure}[H]
	\large
	\begin{tikzpicture}[circuit ee IEC, scale = 1.5]
	%Erweiterung U2	 Messung    	
		\draw [circuit symbol unit = 20pt]
			(2,0) 
			to[resistor = {info = $R_\mathrm{shunt}$, info' = \SI{1}{\kilo\ohm}}] (0,0)  	
			to[voltage source] (0,-2.5)
			-- (2,-2.5);
		\draw [blue,-stealth,line width = 2pt] 
            (-0.75,-0.25) -- (-0.75,-2.25) node[midway,left]{$\underline{U}_\mathrm{E}$};

	%Grundschaltung
		\draw [circuit symbol unit = 20pt]
			(2,0) 
			to[resistor = {info = $\underline{Z}_\mathrm{1}$}] (6,0) 
			to[resistor = {info = $\underline{Z}_\mathrm{2}$}] (8,0) -- (9,0)
			(5.5,0) to[resistor = {info = $\underline{Z}_\mathrm{3}$}] (5.5,-2.5)
			-- (2,-2.5) -- (9,-2.5);
				
		\draw [fill = white] (2,0) circle (1.5pt)
		      		      (9,0) circle (1.5pt)
		      		      (2,-2.5) circle (1.5pt)
		      		      (9,-2.5) circle (1.5pt);
		      		      
		\draw [fill = black] (5.5,0) circle (1.5pt)
				      (5.5,-2.5) circle (1.5pt);
				      
		\draw [blue,-stealth,line width = 2pt] 
			(2,-0.25) -- (2,-2.25) node[midway,left]{$\underline{U}_\mathrm{1}$};
		\draw [blue,-stealth,line width = 2pt] 
			(9,-0.25) -- (9,-2.25) node[midway,right]{$\underline{U}_\mathrm{2}$};
		
		\draw [red,stealth-,line width = 2pt]
			(3,0) -- (2,0) node[midway,above]{$\underline{I}_\mathrm{1}$};
	
	\end{tikzpicture}

  \caption{Schaltbild der T-Netzwerke Zweitor bei Erste versuch}
  \label{fig:3_Schaltung}
\end{figure}
    \begin{figure}[H]
	\large
	\begin{tikzpicture}[circuit ee IEC, scale = 1.5]
	%Erweiterung U2	 Messung    	
		\draw [circuit symbol unit = 20pt]
			(7,0) 
			to[resistor = {info = $R_\mathrm{shunt}$, info' = \SI{1}{\kilo\ohm}}] (9,0)  	
			to[voltage source] (9,-2.5)
			-- (7,-2.5);
		\draw [blue,-stealth,line width = 2pt] 
			(9.75,-0.25) -- (9.75,-2.25) node[midway,right]{$\underline{U}_\mathrm{E}$};
	
	%Grundschaltung
		\draw [circuit symbol unit = 20pt]
			(0,0) 
			to[resistor = {info = $\underline{Z}_\mathrm{1}$}] (4,0) 
			to[resistor = {info = $\underline{Z}_\mathrm{2}$}] (6,0) -- (7,0)
			(3.5,0) to[resistor = {info = $\underline{Z}_\mathrm{3}$}] (3.5,-2.5)
			-- (0,-2.5) -- (7,-2.5);
				
		\draw [fill = white] (0,0) circle (1.5pt)
		      		      (7,0) circle (1.5pt)
		      		      (0,-2.5) circle (1.5pt)
		      		      (7,-2.5) circle (1.5pt);
		      		      
		\draw [fill = black] (3.5,0) circle (1.5pt)
				      (3.5,-2.5) circle (1.5pt);
				      
		\draw [blue,-stealth,line width = 2pt] 
			(0,-0.25) -- (0,-2.25) node[midway,left]{$\underline{U}_\mathrm{1}$};
		\draw [blue,-stealth,line width = 2pt] 
			(7,-0.25) -- (7,-2.25) node[midway,right]{$\underline{U}_\mathrm{2}$};
		
		\draw [red,stealth-,line width = 2pt]
			(5.75,0) -- (6.75,0) node[midway,above]{$\underline{I}_\mathrm{2}$};
	
	\end{tikzpicture}

  \caption{Schaltbild der T-Netzwerke Zweitor bei Zweite versuch}
  \label{fig:3_Schaltung}
\end{figure}
    \[\underline{Z}_1=\underline{Z}_{11}-\underline{Z}_{21}\]
    \[\underline{Z}_2=\underline{Z}_{22}-\underline{Z}_{21}\]
    \[\underline{Z}_3=\underline{Z}_{12}=\underline{Z}_{21}\]
    \[\underline{Z}_1=\sagestr{p(n(R_shunt*(Z11-Z21)))}\si{\ohm}\]
    \[\underline{Z}_2=\sagestr{p(n(R_shunt*(Z22-Z21)))}\si{\ohm}\]
    \[\underline{Z}_3=\sagestr{p(n(R_shunt*(Z21)))}\si{\ohm}\]
\begin{flushright}  
\textit{\autorA}
\end{flushright}
%
%------------------------------%
%------ Ende eures Teils ------%
%------------------------------%
%
%
%
\subsection{Simulation: Admittanzmatrix}
\label{subsec:4_simulation}
%
%------------------------------%
%----- Beginn eures Teils -----%
%------------------------------%
%
Wir wählen eine beliebige Zeitpunkt, womit wir die Admittanzmatrix bestimmen können. Dafür wählen wir eine beliebige index, mit eine Beliebige simulierte Zeitpunkt, wie z.B $t=42,577$.
Dann füllen wir so eine Tabelle
\begin{table}[H]
\begin{tabular}{l|lll|lr} 
\toprule
Messung & $t$ in $\si{\micro\second}$ & $u_1(t)$ in $\si{\volt}$ & \textcolor[rgb]{0,0.102,0.118}{\textbf{\textbf{$u_2(t)$~}}}\textcolor[rgb]{0,0.102,0.118}{\textbf{\textbf{in V}}} & \textcolor[rgb]{0,0.102,0.118}{}$i_1(t) $\textcolor[rgb]{0,0.102,0.118}{in mA} & $i_2(t)$\textcolor[rgb]{0,0.102,0.118}{in mA}  \\ 
\hline 
\textcolor[rgb]{0,0.102,0.118}{Kurzschluss am Ausgang} & 42{,}5527                                                                                                                                                                          & 4{,}5193                                                                                         & 0                                                                                                                 & 12,1730                                                                        & 25.120080                                      \\
\textcolor[rgb]{0,0.102,0.118}{Kurzschluss am Eingang} & 42,5527                                                                                                                                                                          & 0                                                                                              & 4,5193                                                                                                            & -3,6102                                                                        & \multicolumn{1}{l}{4,2601}                     \\
\bottomrule
\end{tabular}
\end{table}
%
\[
 \left(
\begin{array}{cc}
 Y_{11} & Y_{12} \\
 Y_{21} & Y_{22} \\
\end{array}
\right)=\left(
\begin{array}{cc}
 \frac{\text{I}_{1,1}}{U_{1}} & \frac{\text{I}_{1,2}}{U_{2}} \\
 \frac{\text{I}_{2,1}}{U_{1}} & \frac{\text{I}_{2,2}}{U_{2}} \\
\end{array}
\right)\approx \left(
\begin{array}{cc}
2,69356 & -0,798841 \\
 -0,798841 & 0,942845 \\
\end{array}
\right) mS 
\]
\begin{flushright}
\textit{\autorA}
\end{flushright}
%
%------------------------------%
%------ Ende eures Teils ------%
%------------------------------%
%
%
%