%------------------------------------%
%----- 5_Versuchsauswertung.tex -----%
%------------------------------------%
%------------------------------------%
%
%------------------------------%
%----- Beginn eures Teils -----%
%------------------------------%
%
%
%
In diese Simulation haben wir eine Eingangwechselspannung von mit Frequenz, bei den ersten Teil sehen wir, dass wir keine Phasenverschiebung überhaupt, weil wir keine einzige Spule oder Kondensator haben, nur reine Widerstände, wenn wir nur Reelle Impedanzen haben, erwarten wir keine Imaginärteile.

Wir können sehen, bei der Betrachtung der Impedanzmatriz, dass die Komponenten $Z_{12}$ und $Z_{21}$ nach der Theorie gleich seien sollen, aber das ist hier nicht den fall, das kann an der Messabweichung unsere Zeitmessung liegen. Interessanterweise ist die Eingangspannung ein bisschen kleiner im Zweiten teil des Versuchs im vergleich zu ersten Teil, dies könnte darauf zurückgeführt werden, dass bei der Zweite versuch mehr Strom fließt, das könnte verursachen, dass die Signalgenerator die Spannung begrenzt. 

%
\begin{flushright}
  \textit{\autorA}
\end{flushright}
%
%------------------------------%
%------ Ende eures Teils ------%
%------------------------------%
%
%
%