%------------------------------------%
%----- 5_Versuchsauswertung.tex -----%
%------------------------------------%
%------------------------------------%
%
%------------------------------%
%----- Beginn eures Teils -----%
%------------------------------%
%
%
%
In unsere Simulation haben wir eine Eingangwechselspannung von  mit Frequenz, bei den Ersten teil sehen wir, dass wir keine Phasenverschiebung Überhaupt, weil wir keine einzige Spule oder Kondensator haben, nur reine Widerstände, wenn wir nur Reelle Impedanzen haben, erwarten wir keine Imaginärteile.

Wir können sehen, bei der betrachtung der Impedanzmatriz, dass die Komponenten $Z_12$ und $Z_21$ nach der Theorie gleich seien sollen, aber das ist hier nicht den fall, das kann an der Messabweichung unsere Zeitmessung liegen. Interessanterweise unsere Eingangspannung ist ein bisschen kleiner im Zweiten teil des Versuchs im vergleich zu ersten Teil, dies könnte darauf zurückgeführt werden, dass bei der Zweite versuch mehr Strom ausgezogen wird, das könnte verursachen, dass die Signalgenerator die Spannung verkleinert. 

%
\begin{flushright}
  \textit{\autorA}
\end{flushright}
%
%------------------------------%
%------ Ende eures Teils ------%
%------------------------------%
%
%
%