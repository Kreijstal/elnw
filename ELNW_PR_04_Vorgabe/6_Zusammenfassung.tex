%---------------------------------%
%----- 6_Zusammenfassung.tex -----%
%---------------------------------%
%---------------------------------%
%
%------------------------------%
%----- Beginn eures Teils -----%
%------------------------------%
%
%
Bei Analyse komplizierte Schaltungen, können wir versuchen eine Schaltung abstraieren in eine "Allgemeine" n-Pol schaltung, dies gilt nicht für alle mögliche netze mit n-pol Klemmen, ein n-Pol matrize kann zu eine n-Tor matriza umgewandelt werden, es gibt bestimmte Bedingungen (beispielweise der Strom durch die erste Klemme fließt durch die zweite Klemme wieder hinaus), also nicht alle netze lassen sich einfach zu eine Impedanzmatrize Form darzustellen, beispielweise einige netze können sich zu eine Nich invertierende Impedanzmatriz darzustellen, d.h Es gibt keine Admittanzmatriz, analogerweise gibt es Nicht invertierende Admittanzmatrize die nicht zu eine Impedanzmatriz Form kommen können, und es gibt auch Verschiedene Matrizeformen für die Beschreibung ein n-Tor. 

Sei diese Matrize verfügbar, lassen sich unsere rechnungen einfach Simplifizieren, da wir nicht alle Komponenten kennen müssen, sondern nur die $n\times n$ Werte der Matrize, können wir großere teile unsere Netz simplifizieren.
%
\begin{flushright}
  \textit{\autorA}
\end{flushright}
%
%------------------------------%
%------ Ende eures Teils ------%
%------------------------------%
%
%
%