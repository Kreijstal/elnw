%---------------------------------%
%----- 6_Zusammenfassung.tex -----%
%---------------------------------%
%---------------------------------%
%
%------------------------------%
%----- Beginn eures Teils -----%
%------------------------------%
%
%
Bei Analyse komplizierte Schaltungen, können wir versuchen eine Schaltung abstraieren in eine \enquote{Allgemeine} n-Pol schaltung, dies gilt nicht für alle möglichen Netzen mit n-pol Klemmen, ein n-Pol Matrix kann zu eine n-Tor Matrix umgewandelt werden, es gibt bestimmte Bedingungen (beispielweise der Strom durch die erste Klemme fließt durch die zweite Klemme wieder hinaus), also nicht alle Netzen lassen sich einfach zu eine Impedanzmatrix Form darzustellen, beispielweise einige Netzen können sich zu eine nicht invertierende Impedanzmatrix darzustellen, d.h Es gibt keine Admittanzmatrix, analogerweise gibt es nicht invertierende Admittanzmatrize die nicht zu eine Impedanzmatrix form kommen können, und es gibt auch Verschiedene Matrixformen für die Beschreibung ein n-Tor. 

Sei diese Matrizen verfügbar, lassen sich unsere rechnungen einfach simplifizieren, da wir nicht alle Komponenten kennen müssen, sondern nur die $n\times n$ Werte die Matrix, können wir großere Teile unsere Netz simplifizieren.
%
\begin{flushright}
  \textit{\autorA}
\end{flushright}
%
%------------------------------%
%------ Ende eures Teils ------%
%------------------------------%
%
%
%