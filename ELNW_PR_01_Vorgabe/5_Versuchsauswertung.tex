%------------------------------------%
%----- 5_Versuchsauswertung.tex -----%
%------------------------------------%
%------------------------------------%
%
%------------------------------%
%----- Beginn eures Teils -----%
%------------------------------%
%
Wir haben eine Nyquist kurve Simuliert von Frequenz 10 Hz bis 100 kHz, Das Impedanz ist abgebildet als eine Senkrechte geradee von dem Punkt  $470\si{\ohm }-i2.34 \si{\kilo\ohm}$ bis $470\si{\ohm }-i2.3\si{\ohm}$.
Admittanz ist in eine Halbkreis abgebildet, mit eine Spitze in $1.0638+1.0638i \si{\milli\siemens}$, die Kurve fängt bei $0$ an und endet bei ungefähr $21 \si{\milli\siemens}$ und geht im Uhrzeigersinn Richtung.
\par In eine Nyquist Plot lässt sich nicht so gut sehen die Phaseverschiebung in Abhängigkeit von Frequenz (Im vergleich zu eine Bode diagramm) aber man kann trotzdem merken, wenn wir eine niedrige Frequenz haben ist unsere Impedanz sehr groß, die Phase wäre $90^{\circ}$, je schneller die Frequenz desto geringer die Phaseverschiebung wird.
\par  Wir können sehr gut der Einfluss eines Kondensators erkennen, weil die Impedanz in Abhängigkeit die Frequenz ist, dass
heißt, wenn die Frequenz niedrig ist, Impedanz ist groß bzw den Kondensator sperrt Strom, und wenn die Frequenz hat eine
große Frequenz, den Kondensator hat eine kleinere Impedanz, also die spielt eine kleinere Rolle
\par  Wir können sehen, dass es immer Messfehler gibt, Fehlerquellen sind Wärmeverluste, draht Widerstand. Kleine Induktivität bzw
Kapazität in Schaltung/Draht, wegen nicht ideale Bauteile und Abrundungen Fehler von unsere Arithmetik. Auch Kalibrierungen der Oscilloscope und beeinflussing von Elektromagnetische wellen kann Rauschen verursachen.

%
%
%
\begin{flushright}
\textit{\autorA}
\end{flushright}
%
%------------------------------%
%------ Ende eures Teils ------%
%------------------------------%
%
%
%