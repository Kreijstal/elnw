%-----------------------------------------------------%
%------------------ 0-2_Layout.tex -------------------%
%-----------------------------------------------------%
%----- Hier werden Kopf- und Fußzeile definiert. -----%
%----- Außerdem können Befehle erstellt werden, ------%
%----- die euch später im Hauptteil das Leben --------%
%----- deutlich einfacher machen können. -------------%
%-----------------------------------------------------%
%-----------------------------------------------------%
%
% Kopf- und Fußzeile
\pagestyle{scrheadings}							% Seitenstil
\setkomafont{pagehead}{\normalfont}
\clearscrheadfoot										% Lösche alle Voreinstellungen.
\automark{section}									% Setze Automark auf aktuellen Abschnitt.
\ihead{\titel}                      % Schreibe Thema in die Kopfzeile (innen).
% \chead{\headmark}								  % Schreibe Abschnittsüberschrift in die Kopfzeile (Mitte).
\ohead{\autorA}					            % Schreibe Namen in die Kopfzeile (außen).
\cfoot{\pagemark}										% Erstelle (zentrale) Seitennummern.

% Hier können z.B. bequem neue Einheiten für das 'siunitx'-Package definiert werden.
% Passt auf, dass ihr dabei keine vorhandenen Einheiten überschreibt!
%
% \DeclareSIUnit{\voltampere}{VA}
% \DeclareSIUnit{\ah}{Ah}
% \DeclareSIUnit{\wh}{Wh}

% Mit Befehlen können z.B. Abkürzungen für langwierige Formeln o.Ä. erstellt werden.
% Passt auch hier auf, dass ihr nichts überschreibt!
%
% \newcommand{\uin}{\ensuremath{\underline{U}_\mathrm{e}}}
% \newcommand{\uout}{\ensuremath{\underline{U}_\mathrm{a}}}

% Mit dem 'tikz'-Package könnt ihr u.a. (Ersatz-)Schaltbilder zeichnen. Damit die einigermaßen gut aussehen, kann man z.B. Vorlagen für Schaltsymbole, Spannungs- und Strompfeile definieren. Bitte lasst die vorgegebenen Einstellungen stehen, weil spätere Protokollvorgaben darauf aufbauen werden!
%
%--- Ordentliche Proportionen für passive Grundelemente ---%
%
\tikzset
{
  resistor IEC graphic/.style =
  {
    circuit symbol open,
    circuit symbol size = width 2.5 height 1,
    shape = rectangle ee,
    transform shape
  },
  var resistor IEC graphic/.style =
  {
    circuit symbol lines,
    circuit symbol size = width 3.6 height 0.8,
    shape = var resistor IEC,
    transform shape,
    outer sep = 0pt,
    cap = round
  },
  inductor IEC graphic/.style =
  {
    circuit symbol lines,
    circuit symbol size = width 3 height .5,
    transform shape,
    shape = inductor IEC,
    outer sep = 0pt,
    cap = round
  },
  var inductor IEC graphic/.style =
  {
    circuit symbol filled,
    circuit symbol size = width 2.5 height 1
    transform shape,
    shape = rectangle ee
  },
  capacitor IEC graphic/.style =
  {
    circuit symbol lines,
    circuit symbol size = width .5 height 1.5,
    transform shape,
    shape = capacitor IEC
  }
}
%
% --- Messgeräte ---%
%
\tikzset
{
  circuit declare symbol = amperemeter,
  set amperemeter graphic =
  {
    draw,
    generic circle IEC,
    minimum size = 10mm,
    info = center:A
  },
  circuit declare symbol = voltmeter,
  set voltmeter graphic =
  {
    draw,
    generic circle IEC,
    minimum size=10mm,
    info = center:V
  }
}
%
%--- Pfeile ---%
%
\tikzset{
  DickerPfeil/.style = {ultra thick, shorten >= #1, shorten <= #1, -{Straight Barb[angle' = 60, scale = 1.5]}},
	UPfeil/.style = {blue, DickerPfeil = #1, font = {\sffamily\itshape}},
  IPfeil/.style = {red, DickerPfeil = #1, font = {\ttfamily\itshape}}
}